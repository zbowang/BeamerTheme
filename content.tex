%-------------------main body-------------------------%
\usepackage[english]{babel}
\usepackage{amsmath,amssymb,amstext} % math
\usepackage{float}
\usepackage{array}           % array table
\usepackage{fancyhdr}        % header footer
\usepackage{graphicx}        % figure
\usepackage{lmodern}
\usepackage{xcolor}
\usepackage{algorithm2e}
\usepackage{booktabs}

\author{Zhibo Wang}
\title{Beamer Theme}
\date{November 26th, 2018}

\AtBeginSection[]{
    \frame{\frametitle{Outline}\tableofcontents[currentsection, 
    subsectionstyle=show/show/shaded]}
}

\begin{document}

    \frame[plain]{\titlepage}
    \frame{\frametitle{Outline}\tableofcontents}

    \section{Introduction}

    \begin{frame}
        \frametitle{Latex and Beamer}
        
        LaTeX is a high-quality typesetting system; 
        it includes features designed for the production of 
        technical and scientific documentation.

        \vspace{0.4cm}

        \pause

        Beamer is a LaTeX class to create powerful, 
        flexible and nice-looking presentations and slides. 
        
        The beamer class is focussed on producing (on-screen) presentations, 
        along with support material such as handouts and speaker notes.
        
    \end{frame}

    \section{Beamer Basic}
    \subsection{Hightlight}

    \begin{frame}
        \frametitle{Block and Alert}
            
        \vspace{-1.1cm}

        \begin{block}{Pythagorean theorem}
            \vspace*{-\baselineskip}\setlength\belowdisplayshortskip{0.6pt}
            $$a^2 + b^2 = c^2$$
            % \vspace*{-\baselineskip}\setlength\belowdisplayshortskip{0.1pt}
            where c represents the length of the hypotenuse and 
            a and b the lengths of the triangle's other two sides.
        \end{block}
        
        \begin{alertblock}{Remark}
            \begin{itemize}
                \item the environment above is \alert{block}
                \item the environment here is \alert{alertblock}
            \end{itemize}
        \end{alertblock}

        \alert{Hightlight} these words are highlighted by $\backslash alert$.
    \end{frame}

    \begin{frame}
        \frametitle{Proof}

        \begin{block}{Pythagorean theorem}
            \vspace*{-\baselineskip}\setlength\belowdisplayshortskip{0.1pt}
            $$a^2 + b^2 = c^2$$
            % \vspace*{-\baselineskip}\setlength\belowdisplayshortskip{0.2pt}
        \end{block}
        
        \vspace{0.4cm}

        \begin{proof}
            \vspace*{-\baselineskip}\setlength\belowdisplayshortskip{0pt}
            \begin{align*}
                &3^2 + 4^2 = 5^2\\
                &5^2 + 12^2 = 13^2
            \end{align*}
            % \vspace*{-\baselineskip}\setlength\belowdisplayshortskip{0pt}
        \end{proof}
        
    \end{frame}

    \subsection{Other Environments}

    \begin{frame}[shrink=15]{Algorithm}
        
        \begin{algorithm}[H]
            \KwData{this text}
            \KwResult{how to write algorithm with \LaTeX2e }
            initialization\;
            \While{not at end of this document}{
                read current\;
                \eIf{understand}{
                go to next section\;
                current section becomes this one\;
                }{
                go back to the beginning of current section\;
                }
            }
            \caption{How to write algorithms
            (copied from \href{https://en.wikibooks.org/wiki/LaTeX/Algorithms}{here})}
            \end{algorithm}
    \end{frame}

    \begin{frame}{More}

        More environments such as

        \begin{itemize}
            \item Definition
            \item lemma
            \item corollary
            \item example
        \end{itemize}
        
    \end{frame}

    \section{Beamer More}

    \subsection{Split Screen}
    
    \begin{frame}{Minipage}
        \begin{minipage}{0.5\linewidth}
            \begin{figure}[h]
                \includegraphics[width=\textwidth]{imgs/pythagorean.jpg}
            \end{figure}
        \end{minipage}%
        \hfill
        \begin{minipage}{0.4\linewidth}
            \begin{itemize}
                \item item
                \item another
                \item more
                \begin{enumerate}
                    \item first
                    \item second
                    \item third
                \end{enumerate}
            \end{itemize}
        \end{minipage}
        
    \end{frame}

    \begin{frame}{Columns}
        \begin{columns}
            \column{0.5\textwidth}
            This is a text in first column.
            $$E=mc^2$$
            \begin{itemize}
            \item First item
            \item Second item
            \end{itemize}
            
            \column{0.5\textwidth}
            \begin{block}{first block}
                columns achieves splitting the screen
            \end{block}
            \begin{block}{second block}
                stack block in columns
            \end{block}
            
        \end{columns}
    \end{frame}

    \subsection{Table}
    



    \begin{frame}{Create Tables}
        \begin{center}
            \begin{table}[!t]  
                % \caption{Three line}
                % \label{table_time}
                \begin{tabular}{ccc}  
                    \toprule   
                    first&second&third\\ 
                    \midrule       
                    1 & 2 & 3 \\ 
                    4 & 5 & 6 \\ 
                    7 & 8 & 9 \\
                    \bottomrule  
                \end{tabular}
            \end{table}
        \end{center}
    \end{frame}

    \section{Conclusion}

    \begin{frame}{End}
        This document just aims to test styles in beamer.
    \end{frame}

\end{document} 