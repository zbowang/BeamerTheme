\documentclass{beamer}

% This file contains some settings , it need to be input by your tex file.

\usepackage[english]{babel}
% TOC(table of content) using ordered list
\setbeamertemplate{section in toc}[sections numbered]
% unordered list using solid point
\setbeamertemplate{itemize item}{$\bullet$}
% set frame title for each page
\setbeamertemplate{frametitle}
  {\vspace{1cm}
   \insertframetitle
   \vspace{-0.5cm}}
% delete original navigation
\setbeamertemplate{navigation symbols}{}
% set footline
\setbeamertemplate{footline}[text line]{%
    \hfill\strut{%
        \scriptsize\sf\color{black!60}%
        \quad\insertframenumber
    }%
    \hfill
}


% define color
%\definecolor{alizarin}{rgb}{0.82, 0.1, 0.26} % a kind of red
%\definecolor{DarkFern}{HTML}{407428} % a kind of green
%\colorlet{main}{DarkFern!100!white} % first setting way(with color newly defined)
%\colorlet{main}{red!70!black} % second setting way(with internal color by setting gradient)
\definecolor{bistre}{rgb}{0.24, 0.17, 0.12} % a kind of black
\definecolor{mygrey}{rgb}{0.52, 0.52, 0.51} % a kind of grey
\colorlet{text}{bistre!100!white} % from now on, `text` implies the color defined
\colorlet{main}{green!50!black}

% set colors to elements, fg is the color itself, bg is the background color, !num! implies gradient
\setbeamercolor{title}{fg=main}
\setbeamercolor{frametitle}{fg=main}
\setbeamercolor{section in toc}{fg=text}
\setbeamercolor{normal text}{fg=text}
\setbeamercolor{block title}{fg=main,bg=mygrey!14!white}
\setbeamercolor{block body}{fg=black,bg=mygrey!10!white}
\setbeamercolor{qed symbol}{fg=main} % frame after proof
\setbeamercolor{math text}{fg=black}


%% use one of the two below
% \colorlet{main}{red!50!black}
% \colorlet{main}{purple}

%-------------------main body-------------------------%

\author{Your Name}
\title{Presentation Title}
\date{January 1, 2018}

\begin{document}

\frame[plain]{\titlepage}

\begin{frame}
\frametitle{Outline}
\tableofcontents
\end{frame}

\section{Page Title}

\begin{frame}
\frametitle{Page Title}

TeX - LaTeX Stack Exchange is a question and answer site for users of TeX, LaTeX, ConTeXt, and related typesetting systems.

\vspace{0.4cm}

unordered list below

\begin{itemize}
\item The first item
\item The second item
\item The third item
\item The fourth item
\end{itemize}

\end{frame}

\section{Display Theorem}

\subsection{first subsection}

\subsection{second subsection}



\begin{frame}
  \frametitle{Display Theorem}
  \begin{theorem}
    $1 + 2 = 3$
  \end{theorem}
  \begin{proof}
    $$1 + 1 = 2$$
    $$1 + 1 + 1 = 3$$
  \end{proof}
\end{frame}

\section{Sample frame title}

\begin{frame}
\frametitle{Sample frame title}
This is a text in second frame.
For the sake of showing an example.

\begin{itemize}
 \item Text visible on slide 1
 \item Text visible on slide 2
 \item Text visible on slide 3
\end{itemize}

\vspace{0.3cm}

another example

\begin{itemize}\itemsep0em
 \item Text visible on slide 1
 \item Text visible on slide 2
 \item Text visible on slide 3
\end{itemize}
\end{frame}

\begin{frame}
  \begin{proof}
    $$
 \frac{1}{\displaystyle 1+
   \frac{1}{\displaystyle 2+
   \frac{1}{\displaystyle 3+x}}} +
 \frac{1}{1+\frac{1}{2+\frac{1}{3+x}}}
$$
$$\int_0^\infty e^{-x^2} dx=\frac{\sqrt{\pi}}{2}$$
\begin{equation} x=y+3 \label{eq:xdef}
\end{equation}
In equation (\ref{eq:xdef}) we saw $\dots$
  \end{proof}
\end{frame}


\end{document} 